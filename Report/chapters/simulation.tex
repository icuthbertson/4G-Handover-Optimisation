\chapter{Simulation Design}
The simulation is a very important part of the project. It is required to provide the basic functionality of a LTE network. For simplicity the simulation was broken down into two main components; the mobile (UE) and the base station (eNodeB). Due to the project revolving around the handover process in LTE it made sense for the two main components on the simulation to be the mobile and base station as it is the mobile the triggers the measurement report and the base station that makes the decision on whether a handover should take place or not. Since the A3 event trigger is the most common I decided that it would be the only trigger implemented in the simulation to reduce the complexity within the simulation.

It was required for the mobile to be able to move freely around a group of base stations. It was decided that this movement should be random because if any machine learning algorithm can handle random movement then it should also be able to handle regimented movement. The movement that the mobile follows is defined by a Mobility Model and the choice of mobility model is explained in Section~\ref{mobility}.

In wireless communications the received signal strength from a transmitter degrades the further away from the transmitter the receiver is. A propagation model can be used to define the way in which the signal strength degrades. The propagation model is very important to the simulation as it will defined how far away from a base station the mobile can be without dropping the call. A comparison and explanation of the choice of propagation model can be seen in Section~\ref{propagation}.
\section{Discrete Event Simulation}
The concept of time is very important in real world simulations. A popular method for creating this concept of time is \ac{DES}. It works on the basis of a scheduler where events, to be processed at a certain time, are passed to the scheduler and the scheduler then passes the event on to be processed. The way that \ac{DES} creates the concept of time is by allowing events passed in the scheduler to have a time that it should be processed at, the scheduler will keep all the events that are currently still to be processed in an ordered list, the scheduler will then "jump" to the time the first event in the list is to be processed at and passes the event on to be processed. The process of "jumping" to the next time that the first event in the list is to be processed at is what gives \ac{DES} the concept of time.

Another advantage to Discrete Event Simulation is that the passing of events from the scheduler not only gives a concept of time but it also allows for a message to be sent with the event to tell a different part of the simulation what to do at a given time. This message can also sent parameters for the other part of the simulation to use as well.

\section{Mobility Model}\label{mobility}
A mobility model defines the way in which an entity will move. For the purposes of the simulation the mobility model used needed to random in nature. After some reason it was decided that mobility model to be used in the simulation would either be the Random Direction or Random Waypoint model. 

The Random Direction Model is defined as follows:
\begin{enumerate}
\item Select a direction randomly between 0 and 355 degrees.
\item Select a random speed to move at.
\item Select a random duration to move for.
\item Move in the selected direction at the selected speed for the selected duration.
\item Repeat until termination.
\end{enumerate}

An illustration of the movement given by the Random Direction Model can be seen in Figure (placeholder).

place holder for image of random direction

The Random Waypoint Model if defined as follows:
\begin{enumerate}
\item Randomly select the co-ordinates for a point within the environment.
\item Select a random speed to move at.
\item Select a random length of time to pause for when the destination is reached.
\item Move towards the selected co-ordinates at the selected speed
\item Pause for the randomly selected length of time.
\item Repeat until termination.
\end{enumerate}

An illustration of the movement given by the Random Waypoint Model can be seen in Figure (placeholder).

place holder for image of random waypoint

It was decided that the Random Direction Model would be used in the simulation because the Random Waypoint Model has the problem that it is possible to select the co-ordinates of a point very close to where you begin and then pause for a long period time. The possibility of that happening isn’t desired within the simulation. Random Direction does not have this problem and it is also possible to set boundaries on the parameters to make sure that a minimum distance is travelled.~\cite{roy2010handbook}

\section{Propagation Model}\label{propagation}
A propagation model defines how the received signal from a transmitter decays the further from the transmitter you are. There are many different models available, all with different functions and purposes. After some research three models were considered; the Okumura-Hata Model, the Egli Model and the Cost231-Hata Model.

The Okumura-Hata model is very popular for simulating transmissions in built up areas. Equations~\ref{eq:okumura},~\ref{eq:oksmall} and~\ref{eq:oklarge} show the formulas for the model.
\begin{equation}\label{eq:okumura}
L_{u}=69.55+26.16log_{10}f-13.82log_{10}h_{B}-C_{h}+[44.9-6.55log_{10}h_{B}]log_{10}d 
\end{equation}
\begin{equation}\label{eq:oksmall}
C_{H}=0.8+(1.1log_{10}f-0.7)h_{m}-1.56log_{10}f
\end{equation}
\begin{equation}\label{eq:oklarge}
C_{H}=
	\begin{cases}
	8.29(log_{10}(1.54h_{M}))^{2}-1.1 & \mbox{if } 150 \leq f \leq 200 \\
	3.2(log_{10}(11.75h_{M}))^{2}-4.97 & \mbox{if } 200 < f \leq 1500
	\end{cases}
\end{equation}
Where:
\begin{itemize}
\item $Lu$ is the path loss ($dB$).
\item $H_{B}$ is the height of the base station antenna $30$ to $200 m$.
\item $H_{R}$ is the height of the mobile antenna $1$ to $10 m$.
\item $f$ is the frequency of the transmission $150$ to $1500 MHz$.
\item $C_{H}$ is the antenna correction factor.
\item $d$ is the distance between the base station and the mobile $1$ to $20 km$.
\end{itemize}

The Egli Model was another model that was considered for the simulation. Equation~\ref{eq:egli} shows the formula for the model.
\begin{equation}\label{eq:egli}
P_{R50}=0.668G_{B}G_{M}[\frac{h_{B}h_{M}}{d^{2}}]^{2}[\frac{40}{f}]^{2}P_{T}
\end{equation}
Where:
\begin{itemize}
\item $P_{R50}$ is the path loss ($dB$).
\item $P_{T}$ is the power if the transmitter ($W$).
\item $G_{B}$ is the absolute gain of the base station antenna.
\item $G_{M}$ is the absolute gain of the mobile antenna.
\item $h_{B}$ is the height of the base station antenna ($m$).
\item $h_{M}$ is the height of the mobile antenna ($m$).
\item $d$ is the distance between the base station and the mobile ($m$).
\item $f$ is the frequency of the transmission up to $3000 MHz$.
\end{itemize}

The Cost231-Hata model is an extension of the Okumura-Hata to work for frequnencies between 1.5 GHz and 2 GHz. The formulas for this model can be seen in Equations~\ref{eq:cost},~\ref{eq:costahr} and~\ref{eq:metro}.
\begin{equation}\label{eq:cost}
L=46.3+33.9logf-13.82log{h_{B}}-a(h_{R})+[44.9-6.55log{h_{B}}]log{d}+C
\end{equation}
\begin{equation}\label{eq:costahr}
a(h_{R})=(1.1log{f}-0.7)h_{R}-(1.58log{f}-0.8)
\end{equation}
\begin{equation}\label{eq:metro}
C=
	\begin{cases}
	0 dB & \mbox{for medium cities and suburban areas} \\
	3 dB & \mbox{for metropolitan areas}
	\end{cases}
\end{equation}
Where:
\begin{itemize}
\item $L$ is the path loss ($dB$).
\item $f$ is the frequency of the transmission $1500$ to $2000 MHz$.
\item $h_{B}$ is the height of the base station antenna $30$ to $200 m$.
\item $h_{M}$ is the height of the mobile antenna $1$ to $10 m$.
\item $d$ is the distance between the base station and the mobile $1$ to $20 km$.
\item $a(h_{R})$ is the antenna correction factor.
\end{itemize}

After comparing the three models above it was decided that the Cost231-Hata Model would be the one used in the simulation due to it working with frequencies up to $2000 MHz$ (which is the minimum operating frequency of LTE) unlike the Okumura-Hata model which only works up to $1500 MHz$. The Cost231-Hata Model was also picked over the Egli Model because the Egli Model used more parameters that would add more complexity to the simulation.~\cite{chebil2011comparison, shabbir2011comparison}

\section{Simulation Testing}