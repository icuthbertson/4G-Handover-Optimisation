\chapter*{Abstract}
\addcontentsline{toc}{chapter}{Abstract}
With more and more customers using mobile communications it is important for the service providers to give their customers the best \ac{QoS} they can. Many providers have taken to improve their networks and make them more appealing to customers. One such improvement that providers can give to their customers is to improve the reliability of their network meaning that customers calls are less likely to be dropped by the network.

This dissertation explores improving the reliability of a 4G network by optimising the parameters used in handovers. The process of handover within mobile communication networks is very important and allowing for users to move around freely while still staying connected to the network. The parameters used in the handover process are the \ac{TTT} and \ac{hys}. These parameters are used to determined where a base station better then the serving base station by enough to warrant a handover taking place. The challenge in optimising the handover parameters is that there is a fine balance that needs to be struck between calls being dropped due to a handover failing and the connection switching back and forth between two base stations, unnecessarily, wasting the networks resources. The approach taken is to use a machine learning technique known as Q-Learning to optimise the handover parameters by generating a policy that can be followed to adjust the parameters as needed. It was found that the Q-Learning algorithm implemented was capable of improving the performance of the handovers and had the potential of performing even better.
\pagebreak