\chapter{Conclusions}\label{conclusion}
Over the course of this dissertation it has been seen that all of the objectives have been met. The handover parameters, TTT and hys, used in LTE were researched and their functions within the handover process were understood. After that a basic simulation of a LTE network was created with the required functions of handovers, call dropping and connection ping-ponging. The objective met was to implement a machine learning algorithm that would optimised the TTT and hys parameters to improve the performance of the network by reducing the number of dropped calls and connection ping-pongs. The final objective was also met and it was to evaluate the machine learning algorithm to see if it was able to improve the performance of the network.

From the results seen in Section~\ref{handover parameter optimisation} show that the Q-Learning system used was able to improve the performance of the simulated network as a whole. It was seen that the system was able to perform better, over an average of 10 runs, than if the system was not used at all. However, some problems were seen with the system as it was seen to get stuck between states that could have been assumed to be non-optimal, as there were very frequent oscillations between these states. It was also seen that the system may not be the fastest learner as after it has been given a length of time to learn the environment it can only learn as quickly as the frequency of dropped calls and connection ping-pongs.

Overall the project was enjoyable and the majority of the objectives were finished according to the schedule seen in Table~\ref{tab:sch}. The only things that were finished after the end of their given time were the research of the handover parameters and the implementation of network functions, which ended up running over the end of Semester 1, but were finished before the start of Semester 2. The final report was also started behind schedule, being started in week 2 Semester 2.

Throughout the project I have gained a deeper knowledge of both C++ and MATLAB which I have had very little interaction with up until this point. I also feel that this knowledge will be very useful to me in the rest of my university career and beyond. I have gained more knowledge and experienced of machine learning and its practical applications, which had been an interest of mine for some time.
\pagebreak